\documentclass{article}
\usepackage[utf8]{inputenc}
\usepackage{amsmath,amssymb}

\begin{document}
\begin{itemize}

\item 
\textbf{Epistemic Logic Program.}
Let $\mathcal{A}$ be a set of atoms,
let $\mathcal{K(A)}$ be the set $\{ ka \mid a \in \mathcal{A}\}$, 
and assume that $\mathcal{A}$ and $\mathcal{K(A)}$ are disjunct.
%
An epistemic logic program $P$ over $\mathcal{A}$
is a logic program over $\mathcal{A}\cup\mathcal{K(A)}$
such that all atoms occurring in the heads of the rules of 
the program belong to $\mathcal{A}$.
%
A belief view $W$ of $P$ is a set of subsets of $\mathcal{A}$.
 
Let $P$ be an epistemic logic program over $\mathcal{A}$, 
and $W$ be a belief view of $P$.

\item
\textbf{G91 Reduct.} 
The G91 Reduct of $P$ wrt. $W$, written $G91(P,W)$, 
is the result of replacing in $P$
every atom $ka$ such that $a \in K$ by $\top$, and
every other occurrence of the atom $ka$ by $\bot$, 
where $K=\bigcap_{X \in W}X$.

\item
\textbf{G11 Reduct.} 
The G11 Reduct of $P$ wrt. $W$, written $G11(P,W)$, 
is the result of replacing in $P$
every positive literal $ka$ such that $a \in K$ by $a$, and
every negative literal $not \ ka$ such that $a \in K$ by $\bot$, and
every other occurrence of the atom $ka$ by $\bot$, 
where $K=\bigcap_{X \in W}X$.

\item
\textbf{F(oundedness) Reduct.} 
Let $>$ be an total order over the atoms in $\mathcal{A}$.
%
The F Reduct of $P$ wrt. $W$ and $>$, written $F(P,W,>)$, 
is the result of replacing in $P$
every positive literal $ka$ such that $a \in K$ by 
$a \wedge \bigwedge_{b\in\mathcal{A},b < a}b$, and
every negative literal $not \ ka$ such that $a \in K$ by $\bot$, and
every other occurrence of the atom $ka$ by $\bot$, 
where $K=\bigcap_{X \in W}X$.

\item
\textbf{G91-World view.} 
$W$ is a G91-world view of $P$ if
$W$ coincides with the set of stable models of $G91(P,W)$.

\item
\textbf{F-World view.} 
$W$ is a F-world view of $P$ if
$W$ coincides with the set of stable models of $G91(P,W)$, and
there is some order $>$ over $\mathcal{A}$ such that
$W$ is a subset of the stable models of $F(P,W,>)$.

\item
\textbf{G-World view.} 
$W$ is a G-world view of $P$ if
$W$ coincides with the set of stable models of $G91(P,W)$, and
$W$ is a subset of the stable models of $G11(P,W)$.

\item
\textbf{Main Theorem from FAEEL Paper}.
Given any program P, its equilibrium world views coincide
with its founded G91-world views.

\item
\textbf{Hypothesis 1}.
Given any program P, its equilibrium world views coincide
with its F-world views.

\item
\textbf{Hypothesis 2 (wrong)}.
Given any normal program P, its equilibrium world views coincide
with its G-world views.

\item
\textbf{Hypothesis 2 is wrong}.
Consider the program $P$:
\begin{verbatim}
{a}.
b :-  a. c :- not a.
c :- kb. b :- kc.
\end{verbatim}
$W=[\{b,c\},\{a,b,c\}]$ is a G-world view but not an
equilibrium world view.
%
Note that it is also not a F-world view. 
%
For $b < c$ the program $F(P,W,>)$ is:
\begin{verbatim}
{a}.
b :-  a. c :- not a.
c :- b. b :- b, c.
\end{verbatim}
whose stable models are $\{c\}$ and $\{a,b,c\}$, 
so $\{b,c\}$ is not a stable model of $F(P,W,>)$.
%
The case where $b > c$ is symmetrical.

\item
\textbf{Implementation using G-World views 
(useless given that Hypothesis 2 is wrong)}.
It is not difficult to extend the checker to 
verify that all stable models of the check program 
are also stable models of $G11(P,W)$.
%
The verification can also be done outside the current algorithm, 
checking every G91-world view. 

\item
\textbf{Implementation using F-World views}.
In this case we also would have to extend the 
guesser to guess an order of the atoms.

\item
\textbf{Translation to Temporal Logic Programs.}
%
Let $P$ be an epistemic logic program.
%
The temporal version of $P$, written $P(t)$, 
is the temporal logic program that consists of the rules
\[\{\square r \mid r \in P\}\] 
together with the rules
\[\{(next(\square a)) \supset \square(ka) \mid a \in \mathcal{A}\}\].

\item
Let $X$ be a set of atoms, then
$\phi(X)$ denotes the propositional formula $\bigwedge_{a \in X}a \wedge \bigwedge_{a \in \mathcal{A}\setminus X}\lnot a$. 

\item
\textbf{Hypothesis 3}.
Given any normal program P, $W$ is an equilibrium world view of $P$ if, and only if, these conditions hold:
\begin{enumerate}
    \item For every $X \in W$, $P(t)$ bravely entails 
    \[\alpha \wedge \phi(X)\]
    \item $P(t)$ cautiously entails 
    \[\alpha \supset \bigvee_{X \in W}\phi(X)
    \]
\end{enumerate}
where $\alpha$ is
\[next\big(\square(\bigvee_{Y \in W}\phi(Y)) \wedge \bigwedge_{Y \in W}\lozenge\phi(Y)\big).\] 

\item
\textbf{Note.} 
I did not think about empty world views, could they break something?

\item
\textbf{Implementation with Jorge's idea.}
In the guesser, use an ASP version of the program that consists of the rules
\[\{\square r \mid r \in P\}\] 
together with the rules
\[\{(\square a) \supset \square(ka) \mid a \in \mathcal{A}\}.\]
%
Jorge thinks (and to me this makes sense) that we only have to consider as many steps as different subjective atoms occur in the program.

\item
This last point is incorrect, we need to consider as many steps as
stable models in the world view.
Consider the program:
\begin{verbatim}
{a}.
b :- kb, not a.
b :-         a.
\end{verbatim}
$W=[\{b\},\{a,b\}]$ is a G91-World view but 
not an equilibrium world view
(the atom $b$ in $\{b\}$ cannot be justified).
%
The temporal program has a trace with one step $\{a,b\}$.
%
Writing the corresponding temporal program in ASP we would have:
\begin{verbatim}
{a1}.
b1 :- kb1, not a1.
b1 :-          a1.
kb1 :- b1.
\end{verbatim}
that has the stable model $\{a1,b1,kb1\}$.
%
A more general example would be:
\begin{verbatim}
{a(1..100)}.
b :- kb, #false: a(X).
b :-             a(X).
\end{verbatim}
where to show that b is not known we would need 
to consider $2^{100}$ steps.

\item 
\textbf{Hypothesis 4 (One formalization of Jorge's idea).}
A G91-world view $W=\{w_1, \ldots, w_n\}$ is founded iff
$\langle w_1, \ldots, w_n \rangle$ is a 
finite trace of the temporal logic program of Jorge's idea.
\end{itemize}


\end{document}


